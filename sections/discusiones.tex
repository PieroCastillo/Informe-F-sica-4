\documentclass[../main.tex]{subfiles}

\usepackage{float}

\begin{document}
Al obtener el error relativo:
\begin{equation*}
  \epsilon_R (\Delta E) = \frac{4,7}{1,2} \approx 3,92 = \qty{392}{\percent}
\end{equation*}
Como se observa, la precisión del resultado es demasiado baja, tanto que la información que proporciona carece de utilidad. Asimismo, al comparar el valor del trabajo y el de la energía, se identifica que hay una gran diferencia entre ambos, incluso si se intentara interceptar sus incertidumbres, no llegarían a tener ningún valor en común. Un punto importante, es que se observa por el signo negativo, que hubo una pérdida de energía en el experimento, y que, además, es pequeña, esto tiene sentido dado que en la vida real existen fuerzas no conservativas como la fricción con las superficies no solo del plano, sino de los materiales implicados, por ejemplo, al soltar  el aire de la válvula  este ejerce una presión que no eliminara  toda la fricción de la superficie por más que a simple vista lo parezca, también influye la resistencia del aire, deduciendo así que, mientras mayor sea la trayectoria recorrida, la magnitud del valor de la variación de la energía será mayor.
Es muy probable que la causa de esta diferencia entre los valores del trabajo y la variación de energía se encuentre en el hecho de que, en el trabajo se empleó la constante del resorte, y en la variación de energía, ciertamente no, se explica esto porque la constante del resorte B tenía no solo problemas de precisión, también se presume que pudo haber sido mal medida.

Se debe mejorar la precisión al medir los ángulos y las longitudes de las magnitudes de los vectores posición con respecto a los centros A y B.
Asimismo, la medición de los resortes en un tema importante a considerar, pues se presume que esas mediciones se encuentra el responsable de la mayoría de la propagación del error.
\end{document}
