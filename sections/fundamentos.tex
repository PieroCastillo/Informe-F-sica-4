\documentclass[../main.tex]{subfiles}

\begin{document}

\subsection{Trabajo (W)}
Se define como la integral de la fuerza con respecto a la posición:
\[ W = \int_{r_0}^{r_1} \vec{F} \cdot d \vec{r} \]
Si la fuerza es constante, entonces:
\[ W = \vec{F} \cdot \Delta \vec{r} \]
Usando la definición del producto escalar queda:
\[ W = F \cdot \Delta r \cdot cos(\theta)\]
Donde $\theta$ es el ángulo entre el vector fuerza y el vector posición.\\
Y si tomamos en cuenta que la fuerza es tangente a la trayectoria,
nos da la siguiente ecuación:
\[ W = F \cdot \Delta r \]

\subsection{Energía Cinética ($E_c$)}
Si tenemos en cuenta que:
\[ \vec{F} = m \cdot \frac{d \vec{v}}{dt} \]
YT la reemplazamos en la definición de trabajo:
\[ W = \int_{r_0}^{r_1} m \cdot \frac{d \vec{v}}{dt} \cdot d \vec{r} \]
\[ W = \int_{v_i}^{v_f} m \cdot \vec{v} \cdot d \vec{v} \]
\[ W = \frac{1}{2}mv_f^2 - \frac{1}{2}mv_i^2 \]

La expresión anterior nos indica que sin importar el valor de \(\vec{F}\) y la 
trayectoria seguida por la partícula, el valor del trabajo \textit{W}
siempre es igual a la diferencia de las magnitudes de 
\(\frac{1}{2}mv^2\) evaluadas al comienzo y al final de la
 trayectoria. A esta magnitud \(\frac{1}{2}mv^2 \) se la denomina
\textit{energía cinética}.

\[E_c = \frac{1}{2}mv^2 \]
Donde:
\begin{itemize}
    \item $E_c$ : energía cinética
    \item $m$ : masa
    \item $v$ : rapidez
\end{itemize}

\end{document}