\documentclass[../main.tex]{subfiles}

\usepackage{amsmath}

\begin{document}

\subsection{Trabajo (W)}
Se define cómo:
\begin{equation} \label{work_force}
    W = \sum_k W_k = \sum_k \vec{F_k} \cdot \Delta s_k
\end{equation}
Y cuando los desplazamientos  $\Delta s_k$ son muy pequeños,
el trabajo de convierte en \textit{la integral de la fuerza con respecto a la posición}:
\begin{equation} \label{work}
    W = \int_{r_0}^{r_1} \vec{F} \cdot d \vec{r} 
\end{equation}
Si la fuerza es constante, entonces:
\[ W = \vec{F} \cdot \Delta \vec{r} \]
Usando la definición del producto escalar queda:
\[ W = F \cdot \Delta r \cdot cos(\theta)\]
Donde $\theta$ es el ángulo entre el vector fuerza y el vector posición.\\
Y si tomamos en cuenta que la fuerza es tangente a la trayectoria,
nos da la siguiente ecuación:
\begin{equation} \label{work1}
    W = F \cdot \Delta r 
\end{equation}

\subsection{Energía Cinética ($E_c$)}
Si tenemos en cuenta que:
\begin{equation} \label{fuerza}
    \vec{F} = m \cdot \frac{d \vec{v}}{dt}
\end{equation}
Y reemplazando \ref{fuerza} en \ref{work}:
\[ W = \int_{r_0}^{r_1} m \cdot \frac{d \vec{v}}{dt} \cdot d \vec{r} \]
\[ W = \int_{r_0}^{r_1} m \cdot \frac{d \vec{r}}{dt} \cdot d \vec{v} \]
\[ W = \int_{v_i}^{v_f} m \cdot \vec{v} \cdot d \vec{v} \]
\begin{equation} \label{work2}
    W = \frac{1}{2}mv_f^2 - \frac{1}{2}mv_i^2 
\end{equation}

La expresión anterior nos indica que sin importar el valor de \(\vec{F}\) y la 
trayectoria seguida por la partícula, el valor del trabajo \textit{W}
siempre es igual a la diferencia de las magnitudes de 
\(\frac{1}{2}mv^2\) evaluadas al comienzo y al final de la
 trayectoria. A esta magnitud \(\frac{1}{2}mv^2 \) se la denomina
\textit{energía cinética}.

\begin{equation} \label{kineticenergy}
    E_c = \frac{1}{2}mv^2 
\end{equation}
Donde:
\begin{itemize}
    \item $E_c$ : energía cinética
    \item $m$ : masa
    \item $v$ : rapidez
\end{itemize}

\subsection{Teorema Trabajo - Energía Cinética}
Muestra la relación entre el trabajp $(W)$ y la energía cinética $(\Delta E_c)$.\\
Reemplazando \ref{kineticenergy} en \ref{work2}:
\[W = E_{c_2} - E_{c_1} \]
\begin{equation} \label{work3}
    W = \Delta E_c
\end{equation}

\end{document}