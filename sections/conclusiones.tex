\documentclass[../main.tex]{subfiles}
\begin{document}
\begin{enumerate}
  \item La energía y el trabajo están relacionadas ya que, si se aplica energía a un objeto, este genera trabajo siempre y cuando exista un desplazamiento en el objeto al cual se le aplica energía.
  \item Al aplicar la energía en el objeto y al tener este un movimiento, gana energía cinética con la cual es capaz de seguir el movimiento aun cuando se le deja de aplicar una fuerza, esta energía solamente le durará poco tiempo ya que, como hay pérdida de energía en el objeto, este se detendrá.
  \item En el caso de que no tenga pérdida de energía en el proceso de movimiento, siguiendo el principio de la conservación de la energía potencial del objeto en la posición en la que el objeto se detuvo, debería de ser igual a la energía cinética en la posición de movimiento antes de perder su energía.
  \item Existe un error debido a la mala medición de longitudes, es decir que para que el margen de error entre los cálculos teóricos y experimentales sean lo menor posible se recomienda que las medidas tomadas en el laboratorio sean las mas exactas.
  \item La energía mecánica no se conserva debido a la existencia de fuerzas no conservativas como la fricción.
  \item Al soltar el aire de la válvula, este ejerce una presión que no eliminara toda la fricción de la superficie, por más que así parezca durante el experimento.
\end{enumerate}
\end{document}
