\documentclass[../main.tex]{subfiles}
\begin{document}
 La energía y el trabajo están relacionadas ya que, si se aplica energía a un objeto, este genera trabajo siempre y cuando exista un desplazamiento en el objeto al cual se le aplica energía.
 Al aplicar la energía en el objeto y al tener este un movimiento, gana energía cinética con la cual es capaz de seguir el movimiento aun cuando se le deja de aplicar una fuerza, esta energía solamente le durará poco tiempo ya que, como hay pérdida de energía en el objeto, este se detendrá.
 En el caso de que no tenga pérdida de energía en el proceso de movimiento, siguiendo el principio de la conservación de la energía potencial del objeto en la posición en la que el objeto se detuvo, debería de ser igual a la energía cinética en la posición de movimiento antes de perder su energía.
 Existe un error debido a la mala medición de longitudes, es decir que para que el margen de error entre los cálculos teóricos y experimentales sean lo menor posible se recomienda que las medidas tomadas en el laboratorio sean las mas exactas.
 La energía mecánica no se conserva debido a la existencia de fuerzas no conservativas como la fricción.
 Al soltar el aire de la válvula, este ejerce una presión que no eliminara toda la fricción de la superficie, por más que así parezca durante el experimento.

La discrepancia encontrada, se presume que tiene origen en errores sistemáticos, siendo más notorio en la calibración de los resortes ya que, al medir la constante K de los resortes, y en la obtención de resultados se observa que tiene un margen de error considerable, en el caso del resorte B. Además, la perdida de precisión presentada a su vez yace en las aproximaciones consideradas en la guía, tales como la aproximación de la velocidad instantánea con la velocidad media entre dos puntos contiguos y asociar a ese trayecto de los mismos puntos, una fuerza que permanece constante hasta que pase uno de los puntos.
Sin duda, es de suma importancia tener en cuenta el cálculo de incertidumbres al obtener un resultado, puesto que, esto permite compararlos con otras medidas y verificar si realmente la medida es fiable, y, en segundo lugar, qué tan cercano se encuentra al valor exacto.
Se concluye también, que existe un error debido a la mala medición de longitudes, es decir que para que el margen de error entre los cálculos teóricos y experimentales sean lo menor posible se recomienda que las medidas tomadas en el laboratorio sean las más exactas. 
Un punto que sí se pudo comprobar, es que la energía mecánica no se conserva debido a la existencia de fuerzas no conservativas como la fricción, evidenciado en el valor negativo de la variación de energía $\Delta E = \qty{-1,2}{\joule}$.
\end{document}
