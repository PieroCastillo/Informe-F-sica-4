\documentclass[../main.tex]{subfiles}

\usepackage{booktabs}
\usepackage{siunitx}

\begin{document}
\begin{table}[H]
  \caption{Fuerzas sobre el móvil en múltiples puntos de su trayectoria}
  \label{tab:forces}
  \begin{center}
    \begin{tabular}[c]{lrrrrrrrr}
      \toprule
      \multicolumn{1}{c}{\textbf{Punto medio}} &
      \multicolumn{1}{c}{\textbf{Tiempo} (tick)} &
      \multicolumn{1}{c}{\textbf{x_{A}} (\unit{\centi\metre})} &
      \multicolumn{1}{c}{\textbf{x_{B}} (\unit{\centi\metre})} &
      \multicolumn{1}{c}{\textbf{F_{A}} (\unit{\newton})} &
      \multicolumn{1}{c}{\textbf{F_{B}} (\unit{\newton})} &
      \multicolumn{1}{c}{\textbf{F_{A_{t}}} (\unit{\newton})} &
      \multicolumn{1}{c}{\textbf{F_{B_{t}}} (\unit{\newton})} &
      \multicolumn{1}{c}{\textbf{F_{neta_{t}}} (\unit{\newton})} \\
      \midrule
      G & \num{7,5} & \num{15,7} & \num{6,6} & \num{565,2} & \num{217,8} & \num{} & \num{108,9} & \num{} \\
      H & \num{8,5} & \num{17,7} & \num{5,2} & \num{637,2} & \num{171,6} & \num{} & \num{23,88} & \num{} \\
      I & \num{9,5} & \num{19,5} & \num{5,4} & \num{702,0} & \num{178,2} & \num{} & \num{52,10} & \num{} \\
      J & \num{10,5} & \num{20,7} & \num{6,9} & \num{745,2} & \num{227,7} & \num{} & \num{140,2} & \num{} \\
      K & \num{11,5} & \num{21,0} & \num{8,8} & \num{756,0} & \num{290,4} & \num{} & \num{274,6} & \num{} \\
      L & \num{12,5} & \num{20,4} & \num{10,7} & \num{734,4} & \num{353,1} & \num{} & \num{333,9} & \num{} \\
      M & \num{13,5} & \num{18,8} & \num{12,3} & \num{676,8} & \num{405,9} & \num{} & \num{260,9} & \num{} \\
      \bottomrule
    \end{tabular}
  \end{center}
\end{table}
\begin{table}[H]
  \caption{Datos del movimiento del móvil}
  \label{tab:intervals}
  \begin{center}
    \begin{tabular}[c]{lrr}
      \toprule
      \multicolumn{1}{c}{\textbf{Punto medio}} &
      \multicolumn{1}{c}{\textbf{Tiempo (tick)}} &
      \multicolumn{1}{c}{\textbf{Desplazamiento (\unit{\centi\metre})}} \\
      \midrule
      G & \num{7,5} & \num{4,50} \\
      H & \num{8,5} & \num{4,20} \\
      I & \num{9,5} & \num{3,40} \\
      J & \num{10,5} & \num{2,90} \\
      K & \num{11,5} & \num{2,20} \\
      L & \num{12,5} & \num{1,90} \\
      M & \num{13,5} & \num{2,40} \\
    \end{tabular}
  \end{center}
\end{table}

\end{document}
